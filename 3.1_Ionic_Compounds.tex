\documentclass[11pt, letterpaper]{memoir}
\usepackage{HomeworkStyle}
\geometry{margin=0.75in}



\begin{document}

	\begin{center}
		{\large Quiz 3.1 --	Ionic Compounds}
	\end{center}
	{\large Name: \rule[-1mm]{4in}{.1pt} 


\subsection*{Question 1}
Predict which ion each element will form, with the correct charge:

\noindent{\large \ch{Li} \hspace{3em} \ch{O} \hspace{3em} \ch{Sr} \hspace{3em} \ch{Al} \hspace{3em} \ch{Cl} \hspace{3em} \ch{P} \hspace{3em} \ch{K} \hspace{3em} \ch{I} \hspace{3em} \ch{Mg} \hspace{3em} \ch{Ba}}

\vspace{2em}
\subsection*{Question 2}
For each pair of elements, circle the one with higher ionization energy:

\noindent{\large \ch{K} and \ch{Na} \hspace{2em} \ch{N} and \ch{O} \hspace{2em} \ch{Rb} and \ch{Ca} \hspace{2em} \ch{Li} and \ch{Be} \hspace{2em} \ch{Br} and \ch{Cl} \hspace{2em} \ch{Sr} and \ch{Ca}}

\subsection*{Question 3}
For each pair of elements, circle the one with higher magnitude electron affinity:

\noindent{\large \ch{Rb} and \ch{Cs} \hspace{2em} \ch{F} and \ch{O} \hspace{2em} \ch{K} and \ch{Mg} \hspace{2em} \ch{Br} and \ch{Cl} \hspace{2em} \ch{Cl} and \ch{S} \hspace{2em} \ch{Li} and \ch{Na}}

\subsection*{Question 4}
Give the electronic configuration for each of these ions (Hint: Noble gas notation makes these \emph{very} short)

\noindent{\large \ch{S^{2-}} \hspace{6em} \ch{Ca^{2+}} \hspace{6em} \ch{Br^-} \hspace{6em} \ch{Ti^{4+}} \hspace{6em} \ch{N^{-3}} \hspace{6em} \ch{Na^+}}

\vspace{2em}
\subsection*{Question 5}
Give the name for each of these ions 

\noindent{\large \ch{Se^{2-}} \hspace{6em} \ch{Be^{2+}} \hspace{6em} \ch{F^-} \hspace{6em} \ch{Ti^{4+}} \hspace{6em} \ch{N^{-3}} \hspace{6em} \ch{Fe^{2+}}}

\vspace{2em}
\subsection*{Question 6}
Give the symbol for each of these ions 

\noindent{\large Oxide Ion \hspace{1em} Potassium Ion \hspace{1em}  Chromium(III) Ion \hspace{1em} Chloride Ion \hspace{1em} Magnesium Ion \hspace{1em} Vanadium(IV) Ion}

\newpage
\newgeometry{margin=1.25in}
\pagestyle{empty}
\addtocounter{page}{-1}
\section*{\emph{If---}}
\paragraph{By Rudyard Kippling}~
\begin{verse}
	If you can keep your head when all about you\\
	\hspace{0.5em} Are losing theirs and blaming it on you,\\
	If you can trust yourself when all men doubt you,\\
	\hspace{0.5em} But make allowance for their doubting too;\\
	If you can wait and not be tired by waiting,\\
	\hspace{0.5em} Or being lied about, don’t deal in lies,\\
	Or being hated, don’t give way to hating,\\
	\hspace{0.5em} And yet don’t look too good, nor talk too wise:
	
	If you can dream—and not make dreams your master;\\
	\hspace{0.5em} If you can think—and not make thoughts your aim;\\
	If you can meet with Triumph and Disaster\\
	\hspace{0.5em} And treat those two impostors just the same;\\
	If you can bear to hear the truth you’ve spoken\\
	\hspace{0.5em} Twisted by knaves to make a trap for fools,\\
	Or watch the things you gave your life to, broken,\\
	\hspace{0.5em} And stoop and build ’em up with worn-out tools:
	
	If you can make one heap of all your winnings\\
	\hspace{0.5em} And risk it on one turn of pitch-and-toss,\\
	And lose, and start again at your beginnings\\
	\hspace{0.5em} And never breathe a word about your loss;\\
	If you can force your heart and nerve and sinew\\
	\hspace{0.5em} To serve your turn long after they are gone,\\
	And so hold on when there is nothing in you\\
	\hspace{0.5em} Except the Will which says to them: ‘Hold on!’
	
	If you can talk with crowds and keep your virtue,\\
	\hspace{0.5em} Or walk with Kings—nor lose the common touch,\\
	If neither foes nor loving friends can hurt you,\\
	\hspace{0.5em} If all men count with you, but none too much;\\
	If you can fill the unforgiving minute\\
	\hspace{0.5em} With sixty seconds’ worth of distance run,\\
	Yours is the Earth and everything that’s in it,\\
	\hspace{0.5em} And—which is more—you’ll be a Man, my son!
\end{verse}
\end{document}
